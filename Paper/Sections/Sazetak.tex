% !TeX root = ../Document.tex

\documentclass[../Document.tex]{subfiles}

\begin{document}
\section{Sažetak}

Svrha projekta jeste koristeći osnovne dijelove sastaviti i osposobiti robota koji će se balansirati na 2 točka. U projektu su korišteni koračni motori, kao i dva upravljača motorima (A4988). Kako bi se robot balansirao, koristi podatke iz Digital Motion Processor-a koji se nalazi na MPU-6050 te zatim koristeći PID kontrolnu petlju stvara izlaz koji pokreće motore i time vraća robota u balansiran položaj. Smjer i pravac robota je također moguće kontrolisati koristeći jednostavnu Android aplikaciju. Uređaj na kome je instalirana aplikacije se sa robotom povezuje putem Bluetooth konekcije koju robot ostvarju pomoću HC-05 modula.\\

{\Large \noindent \textbf{ Abstract}}\\

\noindent goal of this project is to desgin and create a two-wheeled selfbalancing robot using only basic parts. Parts that were used in this project include two stepper motors and two stepper motor drivers (A4988). The robot balances itself by using data from Digital Motion Processor, which can be found on the MPU-6050,as input in its PID control loop that generates the output used to move the motors whose force balances the robot. The direction in which the robot moves can also be controlled using a simple Android app. The device that has the app installed on it connects to the robot via bluetooth. This is made possible due to the robot having a bluetooth module (HC-05) installed on it.


\end{document}