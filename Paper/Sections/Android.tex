% !TeX root = ../Document.tex
\documentclass[../Document.tex]{subfiles}

\begin{document}
\section{Android aplikacija}
U ovom dijelu će biti opisana Android aplikacija, koja je dio samog projekta samobalansirajućeg robota.

\subsection{Dizajn}
Pošto će se robot moći kretati u 2 smjera i rotirati također u 2 smjera, mobilnu aplikaciju će činiti 4 dugmeta smještena na sredini ekrana.

\cFigure{Android_Dizajn}{Dizajn mobilne aplikacije u Adobe XD}{0.4}

\noindent Dugmići inicijalno imaju postavljenu providnost na 50\%, a vidljivi su 100\% samo dok su pritisnuti.

\subsubsection{GridLayout}
Kako bi dizajn bio responsivan i proporcije ostale jednake na većim i manjim ekranima korišten je GridLayout. Kada se širina layout-a postavi na \verb|match_parent|, mreža će zauzeti širinu uređaja. U ovom slučaju nam je potrebno 3 kolone i 3 reda

\begin{table}[h]
    \centering
    \def\arraystretch{2}
    \begin{tabular}{ |c|c|c| }
        \hline
        Kolona 0, Red 0 & Kolona 1, Red 0 & Kolona 2, Red 0 \\
        \hline
        Kolona 0, Red 1 & Kolona 1, Red 1 & Kolona 2, Red 1 \\
        \hline
        Kolona 0, Red 2 & Kolona 1, Red 2 & Kolona 2, Red 2 \\
        \hline
    \end{tabular}
    \caption{Izgled 3x3 GridLayout-a}
\end{table}

\begin{code}
    \begin{minted}{xml}
        <GridLayout
        android:layout_width="match_parent"
        android:layout_height="wrap_content"
        android:columnCount="3"
        android:rowCount="3">
        <!-- Sadržaj GridLayout-a -->
        </GridLayout>
    \end{minted}
    \caption{GridLayout}
\end{code}

\subsubsection{ImageButton}
ImageButton je zapravo obično dugme kojeg umjesto okvira i teksta u njemu vizuelno predstavlja slika. Kako bih napravio dizajnirani izgled, u odgovarajuće kolone i redove sam postavio ImageButton-e koji će služiti kao dugmad za upravljanje robotom.

\cFigure{Android_Dugmici}{ImageButton-i unutar GridLayout-a}{0.5}

\begin{code}
    \begin{minted}{xml}
        <ImageButton
        android:id="@+id/Right"
        android:layout_row="1"
        android:layout_column="2"
        android:rotation="-90"
        style="@style/DirectionButton"/>
    \end{minted}
    \caption{ImageButton za dugme "Desno"}
\end{code}

\vspace{0.5cm}
\noindent Stil za ImageButton je definisan u \verb|style.xml| datoteci. Rotacija slike se može postići postavljanjem atributa rotation iz imenskog prostora "android". Transparentnost se postavlja atributom alpha, također iz imenskog prostora "android". \verb|"Res/drawable"| je putanja direktorija u kojem se nalaze datoteke slika koje se zatim mogu referencirati iz xml datoteke i postaviti kao zadana pozadina ImageButtona koreisteći se background atributom. Atribut gravity postavlja dugme u središte ćelije GirdLayout-a.\\

\begin{code}
    \begin{minted}[breaklines]{xml}
    <style name="DirectionButton" parent="Widget.AppCompat.ImageButton">
        <item name="android:textColor">#00FF00</item>
        <item name="android:background"> @drawable/ic_arrow_drop_down_circle_black_24dp</item>
        <item name="android:alpha">0.5</item>
        <item name="android:layout_width">0dp</item>
        <item name="android:layout_columnWeight">1</item>
        <item name="android:gravity">center</item>
    </style>
    \end{minted}
    \caption{Still ImageButton-a}
\end{code}

\subsection{Kod}
Pri programiranju Android aplikacija progrmaski  kod se može pisati u Java ili Kotlin programskom jeziku. Ja sam za izradu ove aplikacije odlučio koristiti Java jezik.

\subsubsection{Bluetooth servis}
S obzirom na to da će aplikacija sa robotom komunicirati putem bluetooth veze, potrebno je prvo uspostaviti konekciju, a zatim slati podatke putem iste. Kako bi se koristio bluetooth adapter uređaja, potrebno je u AndroidManifest.xml datoteci dodati opciju za zahtjevanje permisija pri prvom pokretanju aplikacije.

\begin{code}
    \begin{minted}[breaklines]{xml}
    <uses-feature android:name="android.hardware.bluetooth" />
    <uses-permission android:name="android.permission.BLUETOOTH" />
    <uses-permission android:name="android.permission.BLUETOOTH_ADMIN" />
    <uses-permission android:name="android.permission.ACCESS_COARSE_LOCATION" />
    <uses-permission android:name="android.permission.ACCESS_FINE_LOCATION"/>
    \end{minted}
    \caption{Zahtjevanje permisija za korištenje bluetooth-a}
\end{code}

\paragraph{BluetoothAdapter} je objekat koji se nalazi u statičkoj memoriji. U kodu ga je moguće dobiti koristeći \verb|BluetoothAdapter.getDefaultAdapter()|. Adapter posjeduje funkciju \verb|enable| koja će na uređaju upaliti bluetooth ukoliko već nije upaljen.

\paragraph{BluetoothDevice} \label{btdev}je tip objekta koji sadrži informacije o uređaju sa kojim želimo komunicirati. Objekat uređaja možemo dobiti iz BluetoothAdapter-a na osnovu njegove MAC adrese koristeći \verb|getRemote| funkciju.

\paragraph{BluetoothSocket} je objekat koji predstavlja konekciju sa drugim uređajem. Pozivajući funkciju \verb|createInsecureRfcommSocketToServiceRecord| nad objektom uređaja kao rezultat ćemo dobiti BluetoothSocket objekat, nakon čega je potrebno pozvati \verb|connect| funkciju na socket objektom kako bi se veza između uređaja mogla uspostaviti. Svaki socket ima dva stream-a podataka: input i output stream. Iz input streama je moguće čitati podatke koji dolaze na adapter, dok se u output stream upisuju podaci koje želimo poslati povezaonm uređaju.

\paragraph{ConnectionThread} je klasa koju sam napravio kako bih razdvojio bluetooth funkcionalnosti od UI thread-a. Kako bih ovo uradio, bilo je potrebno da ConnectionThread klasa naslijedi Thread roditeljsku klasu. Ova klasa kao parametar u konstruktoru pirma BluetoothSocket čiji outputstream onda smješta u svoju privatnu varijablu. Funkcija kojom se šalju podaci (write) na osnovu String inputa generiše niz byte-ova i koristeći \verb|write| nad socket objektom poruku šalje povezanom uređaju.

\subsubsection{MainActivity}
Aktivnost koja se pokreće pri praljenju aplikacije se naziva "MainActivity". Obzirom da je Java objekto orijentiran programski jezik, aktivnost je javna klasa koja nasljeđuje "AppCompatActivity" klasu.

\begin{code}
    \begin{minted}[breaklines]{java}
    private char _currentSignal;
    private BluetoothService _bluetoothSerivce;
    
    public void SendSignal (char Direction){
        if(_currentSignal==Direction)
            return;
    
        _bluetoothSerivce.Send(Character.toString((Direction)));
        _currentSignal = Direction;
    }
    \end{minted}
    \caption{SendSignal funkcija u MainActivity klasi}
\end{code}

\paragraph{onCreate} je funkcija za koju se očekuje da će biti override-ana od strane programera. Ona se poziva svakiu put kada se aktivnost kreira. Prvo što se treba uraditi unutar onCreate funkcije jeste odabir layout-a koji koristi ova aktivnost pomoću setContentView funkcije.

\paragraph{findViewById(id)} je funkcija koja vraća view objekat na osnovu njegovog id-a koji je postavljen unutar \verb|xml| datoteke.

\paragraph{Eventi}  su objekti koji nastaju kada se desi promjena na korisničkom sučelju. Eventima se mogu dodijeliti funkcije koje će se izvršiti u trenutku kada se event dogodi. Ovi tipovi objekata se najčešće koriste zajedno sa eventListener objektima kojima se odabranim elementima korisničkog sučelja, pridodaju pojedini eventi. Kada se aktivira listener, unutar funkcije se može pristupiti dva objekta, view sa kojim je izvršena interakcija, i tip eventa koji se dogodio. \verb|setOnTouchListener| kao tip eventa može imati \verb|MotionEvent.ACTION_UP|, što označava da dugme više nije u pritisnutom stanju.

\begin{code}
    \begin{minted}[breaklines, tabsize=2]{java}
    List<ImageButton> Buttons = new ArrayList<>();
    Buttons.add((ImageButton) findViewById(R.id.Left));
    Buttons.add((ImageButton) findViewById(R.id.Up));
    Buttons.add((ImageButton) findViewById(R.id.Down));
    Buttons.add((ImageButton) findViewById(R.id.Right));

    for (ImageButton button : Buttons)
    button.setOnTouchListener(new View.OnTouchListener() {
            @Override
            public boolean onTouch(View v, MotionEvent event) {

                    //Dobavljanje objekta koji je pozvao akciju
                    ImageButton imgBtn = (ImageButton) v;
                    //Stavljanje vidljivosti buttona na 100%
                    imgBtn.setAlpha((float) 1.0);

                    //Dobavljanje id stringa buttona
                    String id = getResources().getResourceName(v.getId());
                    int idLocation = id.indexOf('/');
                    id = id.substring(idLocation +1,idLocation+2);

                    //Slanje signala pomoću bluetootha
                    SendSignal(id.charAt(0));

                    //Da li je dugme prestalo biti pritisnut?
                    if(event.getAction() == MotionEvent.ACTION_UP) {
                            //Varćanje vidljivosti na 50%
                            imgBtn.setAlpha((float) 0.5);
                            //Slanje signala za zaustavljanje
                            SendSignal('S');

                        } return true;}});
    \end{minted}
    \caption{Dodjeljivanje event objekata ImageButtonima}
\end{code}


\end{document}